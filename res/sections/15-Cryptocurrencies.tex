\section{Cryptocurrencies correlation}

Blockchain was specifically designed to work with cryptocurrencies. In fact,
transactions works very well for transfer moneys from a user to another, while
every block works like an archive of transactions, allowing everyone to browse
the transaction history and check every moneys movement.

Miners for every transaction and for every block processed get a reward,
usually in the current cryptocurrency. This kind of reward mitigate attackers to
try to modify the Blockchain database to have an higher income.

Any user hold at least \textit{wallet}, that has a specific address. An
address is a random-generated token 26-35 alphanumeric characters long. Users
can have more than one wallet, and creating a new one it doesn't require any
authentication.

Moneys can be sent to non-existing wallets. This transaction is saved in the
block chain and if someone create a wallet with the exact token it will redeem
the money.

\subsection{Privacy during transactions}

Due the fact that every transaction is broadcast to the network there is no
privacy about the transaction content. Everyone can see how much is sent to
other addresses. Many websites provide functionalities\footnote{A complete
website for search Bitcoin transactions and blocks:
\url{https://blockexplorer.com/}.} that allows anyone to easily search data in a
Blockchain database. The ``real'' privacy that Blockchain offer it's defined as
pseudo-privacy: everyone can see the details of any transaction, but it's very
difficult to know the people behind a payment, because the transfer is
identified only with wallets, that aren't physically linked to someone. There
are different studies \cite{guadamuz15} though that point out how it's
relatively easy for an attacker to discover user information thanks to payments
on online shops. An user could buy some good from an online store with a
cryptocurrency, where he's registered with a username and password, linking
\textit{de facto} his identity with his wallet. Using a
\textit{mixer}\footnote{Mixers are online websites that splits the amount to
send to someone into different wallets, then they make fake transactions in
order obfuscate the sender to eventually send the money to the original
recipient.} could mitigate this problem, making more difficult to rebuild
history transactions for a single user.

\subsection{Account management}

User and account concepts don't exists in the Blockchain implementation.
Therefore there isn't any real account management, and no password. Recovering a
lost wallet is almost impossible, because if the asymmetric keys are lost there
is no way to reclaim the transactions, hence to reclaim the money. This
Blockchain characteristic lead to money being locked-in forever into lost
accounts.

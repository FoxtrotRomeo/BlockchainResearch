\section{Beyond Cryptocurrencies}

Thanks to it's design, Blockchain is finding different implementation. Here
we'll briefly show and discuss about the other applications where Blockchain is
being deployed.

\paragraph*{Internet of Things}

Recently, Blockchain has found an application in the IoT\footnote{IoT stands
for \textit{Internet of Things}} world\cite{politecnico16}. Its characteristic
allows to build a decentralized system where IoT devices can store information,
without the possibility from an attacker to change the data already saved in
the database\cite{politecnico16}.

The main problem of using Blockchain as a P2P database in a IoT environment is
that if the network it's too small the possibility for an attacker of taking
control of the main chain are high. In fact, Blockchain is more vulnerable when
the number of miners is small and when the blockchain is new. A solution to
this problem could be use another and mature blockchain, as the Bitcoin one, to
mitigate this attacks or to change the \textit{proof-of-work} mining algorithm.

% TODO am I sure about this?
\paragraph*{Access Control Manager}

As proposed in \cite{dp15}, Blockchain can act as an access manager to enhance
users privacy in modern services. This could be very useful especially in the
mobile area, where applications can have easy access to personal data.

\paragraph*{Storing records}

Blockchain is also able to store data up to 1Mb \cite{ectel16} inside a
transaction, allowing public administrations to use Blockchain to store
records, such as public reports. For instance in \cite{ectel16} the authors
proposed to save papers inside a Blockchain database, in order to keep it
ordered by timestamps and accessible to everyone, thanks to the P2P networking.
